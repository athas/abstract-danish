\begin{center}
  \textbf{Abstract}
\end{center}
In this thesis, I describe two things I did:
\begin{enumerate*}[label=\arabic*)]
\item a compiler transformation that
implements performant automatic differentiation (AD) for a functional,
data-parallel language and
\item an approach for rank polymorphism in a statically typed language with
  parametric polymorphism and type inference.
\end{enumerate*}

On the AD side of things, a method for efficient reverse mode AD on nested
parallel programs is presented. The approach uses a re-computation-based approach
that eliminates storing program variables for the reverse pass; instead,
variables are recomputed as needed in each new scope. Under this technique,
perfectly nested scopes do not introduce re-computation. This is exploited by
applying a repertoire of compiler transformations to transform code into perfect
nests.
%
The language uses a lexicon of high-level parallel combinators---such as
\texttt{map}, \texttt{reduce}, and \texttt{scan}---to build
parallel-by-construction programs. Rewrite rules to differentiate each
combinator are derived, yielding nested-parallel code which itself consists of
parallel combinators. The resulting parallel code is aggressively optimized
using a suite of general and AD-specific optimizations.
%
An implementation in the Futhark programming language is reported on and
evaluated against existing other modern AD implementations on a suite of
benchmarks, demonstrating competitive performance.

On the rank polymorphism side of things, a mechanism for automatically lifting
functions and replicating function arguments in a static context is
presented. The aim is to capture the programming experience in dynamically typed
array languages like NumPy and APL, which permit rank-polymorphic applications,
while also preserving static typing guarantees. The type system---which supports
parametric polymorphism, higher-order functions, and top-level
let-generalization---determines the minimum number of lifting and replication
operations by generating (and solving) integer linear programs from constraints
generated at function application sites. Key theoretical properties of the
mechanism are given. An implementation of the mechanism within the Futhark
compiler is described and demonstrates the system's practicality.
  \newpage
\begin{center}
  \textbf{Resum\'e}
\end{center}
